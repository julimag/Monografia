\abstract
\setlength{\parskip}{2ex plus 0.5ex minus 0.2ex}
\onehalfspacing
\setlength{\parindent}{40pt}
%\setlength{\parskip}{2ex plus 0.5ex minus 0.2ex}
\vspace{-.85cm}

This monograph deals with the importance of user deficient (or someone knowledgeable of their needs) in agile software development accessible.

Customer participation is a key point in software development with agile methodologies. Having access to the customer is the basis for communication and rapid feedback. The ideal is a user representative, one who can correctly answer all the questions that developers can have, and is enabled to be able to make the right decisions.

Agile approaches to induce building products closer to the true reality of the user. This engagement allows fault detection in a more early, as well as the flexibility of agile processes admits correction prematurely. His continuous collaboration can ensure closer scrutiny and attention, largely when that individual has extensive knowledge of what is necessary and required for the end user.

Once the agile fold the constant participation of the customer / user and this may be a deficient, so obtaining an application that satisfies becomes more likely. The collaboration of someone knowledgeable of the problems faced by the end user of the system is undoubtedly a valuable contribution.

One of the principles of the Agile Manifesto is simplicity. The idea is to prevent complications, devices, extravagances and excesses in favor of a lean process with a focus on natural and what is simple. It is possible to establish a direct relationship between the concept of simplicity and digital accessibility, where the first favors the second.

Once the agile manifesto preaches simplicity and this is able to assist in access for all, we can then recognize a connection between successful methodology agile and accessibility. Moreover, the idea of involving the user in the process of software development as being active and participatory is an idea advocated by agile methodology. Considering the deficient user, this involvement is associated values and principles of agile manifesto, with regard to collaboration with the concepts of accessibility. That is compromising deficient in monitoring the production of the functionality of the application is nothing more than an attitude consistent with the agile world and favorable affordability.

The agile methodologies offer reasons and reasons sufficient to justify its use in accessibility projects. In short, it is feasible to think of the adoption of agile development with a focus on universal access.

The main objective of this paper is to highlight the existence of a natural relationship between agile software development and digital accessibility. It is known that the first fold the continued participation of the client, and the thought of secure second, the participant then becomes an individual aware of the difficulties and needs of the disabled.

This paper is not intended to judge the criteria and models of accessibility evaluation software, even studying the tools for measuring these criteria. The aim is to show how nature can be agile development for web design with affordable, or even universal.

One of the future work instigated by this monograph is the definition of a strategy to support the institutionalization of accessibility-focused agile development environments. As well, the application of a case study related to the concepts covered in an agile project with real involvement deficient(s). Furthermore, another challenge is the formal and systematic adoption of accessibility testing aggregates to agile methodologies.

This monograph is a theoretical study. The theme treated favors the realization of new projects related to digital universal access, stimulates organizations to invest in accessibility and contributes to the awareness of the importance of digital inclusion for all. Moreover, from the point of view of the agile world, this paper highlights how important may be the participation of the exact deficient in software development using agile methodologies, which preach the continued involvement of the client. And the proposal also helps to stimulate and disseminate digital assistive technology.

The challenge is not to care for a specific audience, but making a universal development for all, regardless of a disability. We must seek to understand the business needs of the elderly and the deficients. That is, the ideal is not to adapt an existing product to a particular profile, but rather already producing results for all.

\begin{keywords}

Agile software development, accessibility

\end{keywords}
