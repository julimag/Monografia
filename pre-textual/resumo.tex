\resumo
\setlength{\parskip}{2ex plus 0.5ex minus 0.2ex}
\onehalfspacing
\setlength{\parindent}{40pt}
\vspace{-.85cm}

{\flushleft 

Esta monografia trata da import�ncia do cliente no desenvolvimento �gil de software acess�vel.

A participa��o do cliente � requisito no desenvolvimento de software com metodologias �geis. Seja durante todo o processo de desenvolvimento, seja no final de cada itera��o, o cliente tem papel fundamental no sucesso do projeto. 

Da mesma forma, o envolvimento do cliente � fundamental para o pleno atendimento dos crit�rios de acessibilidade. 

Avaliar a acessibilidade do software com a ajuda do pr�prio deficiente ou algu�m conhecedor de suas necessidades garante maior qualidade ao produto.

A presente monografia � um estudo te�rico. Como trabalho futuro, objetiva-se um estudo de caso de um projeto usando as metodologias �geis e com o envolvimento de deficiente(s). Tal projeto poder� validar as vantagens do desenvolvimeno �gil com a participa��o cont�nua do cliente deficiente.

}

% Palavras-chave do resumo em Portugu�s
\begin{keywords}

Desenvolvimento �gil de software, acessibilidade

\end{keywords}
