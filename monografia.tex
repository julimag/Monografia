\documentclass[classic,oneside]{./templates/ufpethesis}
%
% PACKAGES
%
% \usepackage{algorithmicx}
% \usepackage{algorithm}
% \usepackage{algpseudocode}

% pacotes uteis para a	 escrita de documentos LaTeX escritos na lingua portuguesa traduz o documento (i.e. de Abstract para Resumo, o formato da data,...) enquanto o segundo facilita a escrita de acentos e cedilha
\usepackage[portugues,ruled,vlined]{algorithm2e}

% \usepackage[latin1]{inputenc}

%carrega a hifeniza��o em portugu�s e tamb�m coloca os t�tulos de se��es, data, etc. nesse idioma
\usepackage[brazil]{babel}

% for�a o LaTeX a usar fontes do tipo T1. Isso � necess�rio para que os caracteres acentuados possam ser considerados como um bloco s�, e o LaTeX possa hifeniz�-los: caso contr�rio palavras acentuadas n�o ser�o hifenizadas.
% \usepackage[T1]{fontenc}

% precisa ser carregado por causa de um efeito colateral do pacote T1: se voc� gerar PDFs a partir do arquivo tex as fontes v�o ficar terr�veis. Caso voc� n�o v� gerar PDFs ent�o ele n�o precisa ser acrescentado.
% \usepackage{ae}

\usepackage{amssymb} %simbolos matematicos
\usepackage{fancyhdr} %estilo de pagina mais elaborado
\usepackage{cite} %permite quebra de linha em \cite
\usepackage{multirow} %usar varias linhas em uma coluna da tabela
\usepackage{subfigure} %usar figuras lado a lado com mesma legenda

\usepackage{setspace}
\usepackage{graphicx,url,fancybox}
\usepackage{acronym}
% \usepackage{fancyheadings} %obsoleto

%se o usepackage{hyperref} for declarado antes de \usepackage{alg...} ocorre warning: destination with the same identifier
\usepackage{hyperref}

\linespread{1.3} %espacamento entre linhas de ''um e meio''

\setlength{\headheight}{15pt} %evitar Package Fancyhdr Warning: \headheight is too small

\hypersetup{colorlinks,
   debug=false,
   linkcolor=blue,  %%% cor do tableofcontents, \ref, \footnote, etc
   citecolor=red,  %%% cor do \cite
   urlcolor=blue,   %%% cor do \url e \href
   bookmarksopen=true,
   pdftitle={...},
   pdfauthor={},
   pdfsubject={LaTeX},
   pdfkeywords={LaTeX}
}

%
% IDENTIFICACAO
%
\university{Faculdade 7 de Setembro}
\address{Fortaleza-CE}
\institute{Curso de Especializa��o em Desenvolvimento �gil de Software}
\program{P�s-gradua��o}
\majorfield{Desenvolvimento �gil de Software}

\title{Import�ncia do Cliente no Desenvolvimento �gil com Acessibilidade}
\date{Agosto - 2012}
\author{Juliana Feitosa Magalh�es}
\adviser{Prof. Wagner}
% \coadviser{Prof. Me. ?}


%
% INICIO DO DOCUMENTO
%
\begin{document}

% set headings
\pagestyle{fancy}
\lhead[\fancyplain{}{\footnotesize\thepage}]
     {\fancyplain{}{\nouppercase{\footnotesize\rightmark}}}
\rhead[\fancyplain{}{\footnotesize\leftmark}]
     {\fancyplain{}{\nouppercase{\footnotesize\thepage}}}

%
% CAPA
%
\frontmatter
\frontpage
% 
\presentationpage

% \begin{titlepage}

\vspace{1cm}

\center \normalsize{\textbf{Um Estudo sobre a Import�ncia do Cliente no Desenvolvimento �gil de Software com Acessibilidade}}
\vspace{0.5cm}
\center \normalsize{Juliana Feitosa Magalh�es\\  }

\vspace{2cm}

Monografia submetida a Coordena��o do Curso de Especiliaza��o em Desenvolvimento �gil de Software da Faculdade 7 de Setembro como requisito parcial para a obten��o do grau de Especilista em Desenvolvimento �gil de Software.

\vspace{1.5cm}

\vspace{1.5cm}

\underline{ }\underline{ }\underline{ }\underline{ }\underline{ }\underline{ }\underline{ }\underline{ }\underline{ }\underline{ }\underline{ }\underline{ }\underline{ }\underline{ }\underline{ }\underline{ }\underline{ }\underline{ }\underline{ }\underline{ }\underline{ }\underline{ }\underline{ }\underline{ }\underline{ }\underline{ }\underline{ }\underline{ }\underline{ }\underline{ }\underline{ }\underline{ }\underline{ }\underline{ }\underline{ }\underline{ }\underline{ }\underline{ }\underline{ }\underline{ }\underline{ }\underline{ }\underline{ }\underline{ }\underline{ }\underline{ }\underline{ }\underline{ }\underline{ }\underline{ }\underline{ }\underline{ }\underline{ }\underline{ }
\center {Prof. Me. Albert Schilling Gomes}
\vspace{-0.3cm}
\center {Faculdade 7 de Setembro}
\vspace{-0.3cm}
\vspace{1.5cm}

\underline{ }\underline{ }\underline{ }\underline{ }\underline{ }\underline{ }\underline{ }\underline{ }\underline{ }\underline{ }\underline{ }\underline{ }\underline{ }\underline{ }\underline{ }\underline{ }\underline{ }\underline{ }\underline{ }\underline{ }\underline{ }\underline{ }\underline{ }\underline{ }\underline{ }\underline{ }\underline{ }\underline{ }\underline{ }\underline{ }\underline{ }\underline{ }\underline{ }\underline{ }\underline{ }\underline{ }\underline{ }\underline{ }\underline{ }\underline{ }\underline{ }\underline{ }\underline{ }\underline{ }\underline{ }\underline{ }\underline{ }\underline{ }\underline{ }\underline{ }\underline{ }\underline{ }\underline{ }\underline{ }
\center {Prof. Me. Ciro Carneiro Coelho}
\vspace{-0.3cm}
\center {Faculdade 7 de Setembro}
\vspace{-0.3cm}
\vspace{1.5cm}

\end{titlepage}


%
% DEDICATORIA
%
% \begin{dedicatory}
% Aos meus pais, Ol�vio e Jun�lia
% \end{dedicatory}

%
% AGRADECIMENTOS
%
% \acknowledgements

Texto de agradecimento aqui.

% 
% \begin{epigraph}{ Filipenses 4:13 }
% ``Tudo posso Naquele que me fortalece''
% \end{epigraph}

% RESUMO
\resumo
\setlength{\parskip}{2ex plus 0.5ex minus 0.2ex}
\onehalfspacing
\setlength{\parindent}{40pt}
\vspace{-.85cm}

{\flushleft Esta monografia trata da import�ncia do cliente no desenvolvimento �gil de software acess�vel.

A participa��o do cliente � requisito no desenvolvimento de software com metodologias �geis. Seja durante todo o processo de desenvolvimento, seja no final de cada itera��o, o cliente tem papel fundamental no sucesso do projeto. 

Da mesma forma, o envolvimento do cliente � fundamental para o pleno atendimento dos crit�rios de acessibilidade. 

Avaliar a acessibilidade do software com a ajuda do pr�prio deficiente ou algu�m conhecedor de suas necessidades garante maior qualidade ao produto.}

% Palavras-chave do resumo em Portugu�s
\begin{keywords}

Desenvolvimento �gil de software, acessibilidade

\end{keywords}

% 
% % ABSTRACT
\abstract
\setlength{\parskip}{2ex plus 0.5ex minus 0.2ex}
\onehalfspacing
\setlength{\parindent}{40pt}
%\setlength{\parskip}{2ex plus 0.5ex minus 0.2ex}
\vspace{-.85cm}

Esta monografia trata da import�ncia do cliente para os testes de acessibilidade no desenvolvimento �gil de software. 

A participa��o do cliente � requisito no desenvolvimento de software com metodologias �geis. Seja durante todo o processo de desenvolvimento, seja no final de cada itera��o, o cliente tem papel fundamental no sucesso do projeto. 

Da mesma forma, o envolvimento do cliente � fundamental para o pleno atendimento dos crit�rios de acessibilidade. 

Avaliar a acessibilidade do software com a ajuda do pr�prio deficiente ou algu�m conhecedor de suas necessidades garante maior qualidade ao produto.

\begin{keywords}

Desenvolvimento �gil de software, acessibilidade

\end{keywords}


%
% INDICE
%
\tableofcontents

% ABREVIATURAS
% \chapter*{Lista de Abreviaturas}\label{abrev}
%\vspace{-31}
\begin{acronym}
\acro{RUP}     {-  Rational Unified Process}
\end{acronym}


% LISTA DE FIGURAS E TABELAS
% \listoffigures
% \listoftables

%
% CORPO
%
\mainmatter

% % cabecalho mais elaborado
% \pagestyle{fancy}
% 
% \renewcommand{\headrulewidth}{0.3pt} %largura da linha do cabecalho
% \renewcommand{\footrulewidth}{0.3pt} %largura da linha do rodape
% %\renewcommand{\chaptermark}[1]{\markboth{#1}{#1}} %exibir capitulo
% \renewcommand{\sectionmark}[1]{\markright{\thesection\ #1}}% exibir igual a secao (nao coloca em letra maiuscula!)
% 
% \lhead{\sl\small\rightmark} % exibe a secao, \leftmark exibe o capitulo
% \rhead{}
% \lfoot{\sl\small\titulo} % exibe titulo
% \cfoot[\fancyplain{\thepage}{}]{\fancyplain{\thepage}{}}
% \rfoot{\small\thepage} % exibe numero da pagina

% set footers
% \lfoot{\sl\small\titulo} % exibe titulo - precisa declarar macro titulo
% \cfoot[\fancyplain{\thepage}{}]{\fancyplain{\thepage}{}}
% \rfoot{\small\thepage} % exibe numero da pagina

%===============================================================================
% CAPITULO 1 - INTRODUCAO
%===============================================================================
\chapter{Introdu��o}\label{chp:1}

%===================================================================
\section{Motiva��o e Caracteriza��o do Problema}
%===================================================================

O sucesso de um produto depende principalmente de sua qualidade, mas o cumprimento de prazos e custos estipulados para sua produ��o tamb�m s�o fundamentais. Em se tratando de software, as exig�ncias n�o s�o diferentes. As metodologias de desenvolvimento, seja tradicional ou �gil, buscam obter o resultado com �xito. E esse resultado deve ser preciso e em conformidade com os anseios do cliente.

As metodologias tradicionais d�o �nfase ao processo, ao controle do andamento do projeto e � exist�ncia de documenta��o. Enquanto as metodologias �geis focam as pessoas, a r�pida entrega do produto e a obten��o de respostas, al�m de permitir o cliente participar de forma mais pr�xima e colaborativa.

Para atingir a plena qualidade, um software deveria atender as recomenda��es de acessibilidade, pelo menos quando assim o for poss�vel (exceto, por exemplo, softwares embarcados sem nenhum interface).

A id�ia de um software atender a todos, inclusive pessoas com necessidades especiais, � cada vez mais almejada. Diante disso, este trabalho tem por motiva��o destacar a import�ncia da participa��o do deficiente e/ou algu�m conhecedor de suas necessidades no desenvolvimento do software. Visto que, um dos princ�pios do desenvolvimento �gil � "privilegiar a comunica��o direta, cara a cara", iremos conciliar a id�ia da participa��o do cliente com sua influ�ncia na garantia do cumprimento e atendimento da acessibilidade.

%===================================================================
\section{Objetivo e Contribui��o}
%===================================================================

O objetivo desta monografia � destacar a import�ncia do cliente deficiente ou de algu�m conhecedor das limita��es dos portadores de necessidades especiais durante o desenvolvimento de software. Nesse contexto, a escolha de m�todos �geis se encaixam de forma natural, uma vez que pregam a participa��o do cliente durante o processo.

Esta monografia visa minimizar a falta de conhecimento e aumentar as iniciativas na �rea de acessibilidade digital. O envolvimento do indiv�duo capaz de representar o usu�rio deficiente � apresentado como algo singular no desenvolvimento de software acess�vel. E esse aspecto j� � assegurado pelos m�todos �geis. Sendo assim, unir a id�ia de desenvolvimento de software �gil com a id�ia de acessibilidade torna-se algo perfeitamente apropriado.

%===================================================================
\section{Organiza��o do Texto}%Estrutura do Trabalho
%===================================================================

Esta monografia est� organizada em cinco cap�tulos. No Cap�tulo 1, apresenta-se a motiva��o e caracteriza��o do problema a ser tratado, o objetivo a ser alcan�ado e a contribui��o resultante do desenvolvimento deste trabalho.

O Cap�tulo 2 exp�e a fundamenta��o te�rica relacionada ao tema abordado. S�o apresentadas as caracter�sticas das metodologias de desenvolvimento tradicional e �gil de software. S�o citados os conceitos referentes ao design universal, design acess�vel e tecnologia assistiva digital. S�o apresentados os conceitos de acessibilidade e usabilidade digital, e tamb�m sobre testes de acessibilidade e usabilidade, al�m das diretrizes de acessibilidade na web no mundo e no Brasil.

O Cap�tulo 3 descreve o trabalho propriamente dito. Explica o tema proposto com esclarecimentos particulares para o a abordagem adotada. Ademais, s�o destacados os pontos mais relevantes da monografia.

No Cap�tulo 4 � feita uma an�lise da contribui��o e no Cap�tulo 5 encontra-se a conclus�o final do trabalho.

 %Introducao
%===============================================================================
% CAPITULO 2 - ESTADO DA ARTE
%===============================================================================
\chapter{Fundamenta��o Te�rica}\label{chp:2}

%===================================================================
\section{Metodologias de Desenvolvimento}
%===================================================================

Metodologia de desenvolvimento � um conjunto de pr�ticas recomendadas para o desenvolvimento de softwares, sendo que essas pr�ticas, geralmente, passam por fases ou passos, que s�o subdivis�es do processo para orden�-lo e melhor gerenci�-lo \cite{Sommerville2007}.

%===================================================================
\subsection{Desenvolvimento Tradicional}
%===================================================================

As metodologias tradicionais s�o tamb�m chamadas de pesadas ou orientadas a documenta��o. Essas metodologias surgiram em um contexto de desenvolvimento de software muito diferente do atual, baseado apenas em um mainframe e terminais burros. Na �poca, o custo de fazer altera��es e corre��es era muito alto, uma vez que o acesso aos computadores era limitado e n�o existiam ferramentas modernas de apoio ao desenvolvimento do software, como depuradores e analisadores de c�digo. Por isso, o software era todo planejado e documentado antes de ser implementado. A principal metodologia tradicional e muito utilizada at� hoje � o modelo cl�ssico \cite{Soares2004}.

O modelo cl�ssico ou cascata caracteriza-se pelo seu car�ter preditivo, prescritivo, sequencial, burocr�tico, rigoroso, orientado a processos e dados, formais e controlado, que tem o sucesso alcan�ado desde que esteja em conformidade com o que foi planejado \cite{Mainart2010}.

De uma forma geral, as etapas do modelo cl�ssico s�o: defini��o de requisitos, projeto do software, implementa��o e teste unit�rio, integra��o e teste do sistema, opera��o e manuten��o. Esta divis�o em etapas distintas e certa inflexibilidade dificulta poss�veis altera��es, comuns no desenvolvimento de projetos. Por esse motivo, seu uso � recomend�vel apenas em situa��es em que os requisitos s�o est�veis e os requisitos futuros s�o previs�veis. Al�m disso, quando todas as exig�ncias s�o conhecidas antes do in�cio da fase de desenvolvimento, � mais f�cil definir cronogramas, determinar os custos do projeto e alocar recursos.

%===================================================================
\subsection{Desenvolvimento �gil}
%===================================================================

Em 2001, um grupo de dezessete profissionais veteranos da �rea de software se reuniu para discutir sobre seus trabalhos e os m�todos utilizados. Embora cada um deles tivesse suas pr�prias pr�ticas e teorias de como fazer um projeto de software ter sucesso, cada qual com as suas particularidades, eles imaginavam haver um subconjunto de princ�pios comum. 

A partir do reconhecimento dos aspectos similares, eles criaram o Manifesto para o Desenvolvimento �gil de Software \cite{Manifesto2001}, freq�entemente chamado apenas de Manifesto �gil, e o termo Desenvolvimento �gil passou a descrever abordagens de desenvolvimento que seguissem os tais princ�pios.

O manifesto � composto de quatro valores e doze princ�pios, os quais descrevem a ess�ncia de um conjunto de abordagens para desenvolvimento �gil de software. 

Os valores s�o: 

\begin{enumerate}
 \item Indiv�duos e intera��es ao inv�s de processos e ferramentas;
 \item Software execut�vel ao inv�s de documenta��o;
 \item Colabora��o do cliente ao inv�s de negocia��o de contratos;
 \item Respostas r�pidas a mudan�as ao inv�s de seguir planos.
\end{enumerate}

E os princ�pios s�o:

\begin{enumerate}
 \item Nossa maior prioridade � satisfazer o cliente, atrav�s da entrega adiantada e cont�nua de software de valor;
 \item Aceitar mudan�as de requisitos, mesmo no fim do desenvolvimento. Processos �geis se adequam a mudan�as, para que o cliente possa tirar vantagens competitivas;
 \item Entregar software funcionando com freq�encia, na escala de semanas at� meses, com prefer�ncia aos per�odos mais curtos;
 \item Pessoas relacionadas � neg�cios e desenvolvedores devem trabalhar em conjunto e di�riamente, durante todo o curso do projeto;
 \item Construir projetos ao redor de indiv�duos motivados. Dando a eles o ambiente e suporte necess�rio, e confiar que far�o seu trabalho;
 \item O M�todo mais eficiente e eficaz de transmitir informa��es para, e por dentro de um time de desenvolvimento, � atrav�s de uma conversa cara a cara;
 \item Software funcional � a medida prim�ria de progresso;
 \item Processos �geis promovem um ambiente sustent�vel. Os patrocinadores, desenvolvedores e usu�rios, devem ser capazes de manter indefinidamente, passos constantes;
 \item Cont�nua aten��o � excel�ncia t�cnica e bom design, aumenta a agilidade;
 \item Simplicidade: a arte de maximizar a quantidade de trabalho que n�o precisou ser feito;
 \item As melhores arquiteturas, requisitos e designs emergem de times auto-organiz�veis;
 \item Em intervalos regulares, o time reflete em como ficar mais efetivo, ent�o, se ajustam e otimizam seu comportamento de acordo.
\end{enumerate}


O manifesto reconhece a import�ncia de determinados conceitos como processos, ferramentas, documenta��o, contratos e planos no desenvolvimento de software, mas identifica aspectos ainda mais importantes a serem valorizados.

%===================================================================
\section{Design Universal e Design Acess�vel}
%===================================================================

O conceito de design universal � diferente do conceito de design acess�vel. O design acess�vel diz respeito aos produtos e constru��es acess�veis e utiliz�veis por pessoas com defici�ncias. O design universal diz respeito aos produtos e constru��es acess�veis e utiliz�veis por todos os indiv�duos, independentemente de possu�rem ou n�o defici�ncias. Apesar das defini��es aparentarem ter apenas diferen�as sem�nticas, na realidade significam muito mais do que isto. Os designs acess�veis t�m uma tend�ncia a separar as facilidades oferecidas para as pessoas com defici�ncias, das oferecidas �s demais pessoas, como rampas de acesso ao lado de escadas ou toaletes diferentes para cadeirantes. O design universal, por outro lado, tem como objetivo proporcionar solu��es que possam acomodar pessoas com ou sem defici�ncias e beneficiar pessoas de todas as idades e capacidades, sem discrimina��es \cite{Carvalho2003}.

O design universal pressup�e a acessibilidade f�sica e a acessibilidade virtual (ou digital), sendo a primeira a garantia de mobilidade e usabilidade para qualquer pessoa em todos os espa�os, e a segunda a garantia de mobilidade e usabilidade de recursos computacionais \cite{Saci}.

%===================================================================
\section{Acessibilidade e Usabilidade Digitais}
%===================================================================

Acessibilidade � o termo geral usado para indicar a possibilidade de qualquer pessoa usufruir todos os benef�cios de uma vida em sociedade, entre eles, o uso da Internet; essa defini��o, proposta inclusive pela Associa��o Brasileira de Normas T�cnicas, apesar de forte impacto, � fundamental, pois acessibilidade s� existe quando todos conseguem acessar esses benef�cios \cite{Nunes2008}.

A usabilidade � a caracter�stica que determina se o manuseio de um produto � f�cil e rapidamente aprendido, dificilmente esquecido, n�o provoca erros operacionais, oferece um alto grau de satisfa��o para seus usu�rios e, eficientemente resolve as tarefas para as quais ele foi projetado. Uma aplica��o orientada � usabilidade n�o necessariamente � orientada � acessibilidade, e vice-versa. Ou seja, ela pode ser de f�cil uso para usu�rios comuns, mas inacess�vel para os com necessidades especiais \cite{Nunes2008}.

Por vezes, os conceitos de acessibilidade e usabilidade se confundem. Enquanto a usabilidade volta-se mais para as expectativas e para a capacidade do usu�rio em entender e perceber as estrat�gias de utiliza��o do software, a acessibilidade est� voltada para as condi��es de uso, como o usu�rio se apresenta frente �s interfaces interativas, como essa troca deve acontecer, e, principalmente, como se dar� o acesso do usu�rio �s informa��es dispon�veis \cite{Passerino2007}.

Ser acess�vel � permitir o uso. O fato de estar vis�vel e ser percept�vel n�o garante a condi��o assistiva � interface, mas sim, quando ela considera as necessidades especiais de cada sujeito e cumpre esse requisito. 

A acessibilidade mede-se em termos de flexibilidade do produto para atender �s necessidades e prefer�ncias do maior n�mero de pessoas. Mas isso n�o � suficiente, ele tamb�m deve ser compat�vel com tecnologias assistivas ao viabilizar sua pr�pria adaptabilidade de acordo com as necessidades e demandas dos usu�rios, independente do grau, n�vel ou intensidade de sua necessidade \cite{Passerino2007}.

Acessibilidade e usabilidade s�o alguns dos conceitos que norteiam a qualidade de uso dos sistemas.

Entende-se por acessibilidade � rede a possibilidade de qualquer indiv�duo, utilizando qualquer tipo de tecnologia de navega��o (navegadores gr�ficos, textuais, especiais para cegos ou para sistemas de computa��o m�vel), poder visitar qualquer site e obter um total e completo entendimento da informa��o contida nele, al�m de ter total e completa habilidade de intera��o. A acessibilidade das p�ginas web depende da intera��o de tr�s elementos, quais sejam: os sistemas de acesso ao computador (ajudas t�cnicas), os navegadores utilizados e o desenho das p�ginas que comp�em os sites web \cite{Sonza2008}.

O conceito de qualidade de uso mais amplamente utilizado � o da usabilidade. Tal conceito est� relacionado � facilidade e efici�ncia de aprendizado e de uso, bem como, � satisfa��o do usu�rio. Quando tratamos de usabilidade, h� basicamente duas abordagens fundamentais ao design de ambientes: o ideal art�stico e o ideal de engenharia. Embora a arte seja importante, o principal objetivo da maioria dos projetos da web deve ser o de facilitar aos usu�rios o desempenho de tarefas �teis \cite{Sonza2008}.

Um conceito que come�a a ser utilizado na atualidade � o da usabilidade aplicada � acessibilidade. Tal pr�tica amplia o entendimento de acessibilidade virtual ao mencionar a import�ncia n�o apenas de se aplicar as recomenda��es do W3C, mas tamb�m de se tornar os ambientes f�ceis de usar para todos, ou seja: aplicar usabilidade nos sites para torn�-los verdadeiramente acess�veis \cite{Sonza2008}.

%===================================================================
\section{Tecnologia Assistiva Digital}
%===================================================================

Tecnologia Assistiva (TA) refere-se ao conjunto de artefatos disponibilizados �s pessoas com necessidades especiais, que contribuem para prover-lhes uma vida mais independente, com mais qualidade e possibilidades de inclus�o social \cite{Sonza2008}.

A tecnologia assistiva digital refere-se aos sistemas que oferecem solu��es para tentar suprir as limita��es de uma parcela da sociedade. Essas limita��es podem ser motoras, visuais, auditivas, entre outras.

As tecnologias da informa��o (hardware e software) s�o muitas vezes projetadas sem considerar a diversidade de acesso dos v�rios utilizadores. De fato, muitas pessoas apresentam dificuldades de utiliza��o do teclado, do mouse, do monitor/tela, seja devido a tetraplegia, problemas no controle efetivo das m�os, perda dos membros superiores, paralisia cerebral, cegueira ou baixa vis�o. Assim, ao projetar sistemas de informa��o, deve-se prever uma s�rie de possibilidades/alternativas de acesso (ao n�vel de hardware e de software), contemplando a acessibilidade motora, a acessibilidade auditiva, a acessibilidade visual e a acessibilidade cognitiva \cite{Correia2005}.

Atualmente os pr�prios sistemas operacionais, inclusive �queles para dispositivos m�veis, possuem alguns recursos de acessibilidade embutidos no pr�prio software.

As interfaces com o usu�rio devem poder ser acessadas por qualquer pessoa, independentemente de suas capacidades f�sico-motoras e perceptivas, culturais e
sociais \cite{Nunes2008}.

%===================================================================
\section{Testes de Acessibilidade e Usabilidade}
%===================================================================

A verifica��o de acessibilidade de sites � feita atrav�s de programas que detectam o c�digo e analisam seu conte�do, verificando se est� dentro do conjunto das regras; no final, geram uma lista dos problemas que devem ser corrigidos para que o site seja considerado acess�vel. Destacam-se: WebXact, o Hera e o brasileiro daSilva \cite{Nunes2008}.

Contudo, � importante ressaltar que pessoas com necessidades especiais desenvolvem habilidades espec�ficas. Por exemplo, deficientes visuais usam as combina��es das teclas de tal forma que uma pessoa com vis�o n�o conseguiria simular. Logo, para se obter um site de acesso universal orientado � usabilidade, al�m de verific�-lo atrav�s de programas avaliadores, � fundamental que se considerem as dificuldades e habilidades dos usu�rios, pois estas norteiam o modelo mental de suas intera��es e, ao serem consideradas, contribuem para tornar a intera��o do deficiente harmoniosa \cite{Nunes2008}.

Com rela��o � acessibilidade, os problemas de usabilidade, em geral ocorrem por tr�s motivos: muito foco na conformidade com as diretrizes de acessibilidade e n�o na usabilidade; muitos programas avaliadores dependem somente de t�cnicas de verifica��o sint�tica dos sites para detectar a acessibilidade e, com isso, os erros detect�veis se limitam � camada de descri��o de tags (etiquetas) e n�o consideram aspectos de usabilidade; por fim, os avaliadores de acessibilidade desconsideram o fato que raramente os usu�rios escutam a sa�da falada de forma passiva. Eles se movimentam pelas p�ginas usando combina��es de teclas e, atrav�s desse processo, criam seus modelos mentais \cite{Nunes2008}.

N�o � poss�vel projetar todos os produtos para que sejam utilizados por todas as pessoas, pois sempre haver� algu�m com uma combina��o de graves defici�ncias, que n�o ser� capaz de utiliz�-los. Sendo assim, um produto n�o pode ser caracterizado simplesmente como acess�vel ou n�o acess�vel. Produtos s�o, na verdade, mais ou menos acess�veis, e devem atender (ou, de prefer�ncia, exceder) padr�es m�nimos de acessibilidade fixados por lei ou norma t�cnica \cite{Sonza2008}.

Quando tratamos do mundo digital, a acessibilidade pode envolver tr�s grandes �reas \cite{Sonza2008}:

\begin{enumerate}
 \item acessibilidade ao computador: ajudas t�cnicas que podem ser gen�ricas ou especialmente projetadas para facilitar a tarefa de navega��o � web. Nesse grupo encontram-se tanto os programas (software), como os equipamentos f�sicos (hardware) de acesso;
 \item acessibilidade do navegador utilizado: o programa utilizado para apresentar o conte�do da web ao usu�rio pode ser gen�rico, como o Windows Explorer, Netscape Navigator, Mozilla Firefox, ou espec�fico, que oferece facilidades de acesso a determinados grupos de usu�rios, como � o caso do navegador Lynx para usu�rios cegos;
 \item acessibilidade ao desenvolvimento de p�ginas web: nesse ponto, � importante que haja distin��o entre conte�do e apresenta��o de cada p�gina; para isso, torna-se imprescind�vel utilizar uma ferramenta que ofere�a maiores funcionalidades e op��es para a cria��o de ambientes acess�veis.
\end{enumerate}

Acessibilidade � Internet refere-se ao desenvolvimento de p�ginas que estejam dispon�veis e acess�veis via web a qualquer hora, local, ambiente, dispositivo de acesso e por qualquer tipo de visitante/usu�rio. Uma interface que n�o � acess�vel a uma determinada pessoa, n�o poder� ser considerada eficaz, eficiente ou mesmo agrad�vel a ela. A acessibilidade relaciona-se tamb�m ao contexto de uso, necessidades e prefer�ncias dos usu�rios t�picos \cite{Sonza2008}.

 %Fundamentacao Teorica

\backmatter

%
% BIBLIOGRAFIA
%

%estilos: abbrv, acm, alpha, apalike, ieeetr, plain, siam e unsrt
\bibliographystyle{abnt-alf} %ordenado por [nome_autor] ao inves de [numero]
\bibliography{monografia}

\end{document}