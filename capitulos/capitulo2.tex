%===============================================================================
% CAPITULO 2 - ESTADO DA ARTE
%===============================================================================
\chapter{Fundamenta��o Te�rica}\label{chp:2}


%===================================================================
\section{Metodologias �geis}
%===================================================================

O termo Metodologias �geis tornou-se popular em 2001 quando dezessete especialistas em processos de desenvolvimento de software representando os m�todos Scrum \cite{Scrum2001}, Extreme Programming (XP) \cite{XP2004} e outros, estabeleceram princ�pios comuns compartilhados por todos esses m�todos. Foi ent�o criada a Alian�a �gil e o estabelecimento do Manifesto �gil \cite{Manifesto2001}.

Os conceitos chave do Manifesto �gil s�o: 

\begin{enumerate}
 \item Indiv�duos e intera��es ao inv�s de processos e ferramentas;
 \item Software execut�vel ao inv�s de documenta��o;
 \item Colabora��o do cliente ao inv�s de negocia��o de contratos;
 \item Respostas r�pidas a mudan�as ao inv�s de seguir planos.
\end{enumerate}

O manifesto reconhece a import�ncia de determinados conceitos como processos, ferramentas, documenta��o, contratos e planos no desenvolvimento de software, mas identifica conceitos ainda mais importantes a serem valorizados.

%===================================================================
\section{Metodologias Tradicionais}
%===================================================================

%===================================================================
\section{Testes de Acessibilidade}
%===================================================================
