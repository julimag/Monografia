%===============================================================================
% CAPITULO 3 - Metodologia
%===============================================================================
\chapter{Metodologia}\label{chp:3}

%===================================================================
\section{Caracteriza��o do Estudo}
%===================================================================

Para o desenvolvimento desta monografia, foram seguidos os seguintes passos:
\begin{enumerate}
 \item O estudo foi dividido permitindo a contextualiza��o de sub-temas espec�ficos que comp�em o assunto, atrav�s de pesquisa bibliogr�fica, e que s�o:
 \begin{enumerate}
  \item Conceitos pertinentes ao desenvolvimento tradicional de software;
  \item Conceitos pertinentes ao desenvolvimento �gil de software;
  \item Acessibilidade digital.
 \end{enumerate}
 \item Com base no entendimento dos sub-temas foi poss�vel concluir a import�ncia da jun��o de alguns aspectos. E assim, foi poss�vel destacar, ainda mais, a relev�ncia da participa��o do cliente no mundo do desenvolvimento digital com contribui��o social.
\end{enumerate}

%===================================================================
\section{Ambiente Web}
%===================================================================

%===================================================================
\section{Deficiente Visual}
%===================================================================

%===================================================================
\section{Desenvolvimento �gil com Acessibilidade Digital}
%===================================================================

Apesar de j� existir diversos elementos auxiliares para facilitar a inclus�o digital de pessoas com certas limita��es, � preciso termos consci�ncia da import�ncia de considerar a condi��o restritiva de cada um. O desenvolvimento de software acess�vel n�o deve se resumir apenas a atender uma lista de recomenda��es de um dado modelo de acessibilidade. Os itens definidos nos modelos existentes s�o meras instru��es para orientar o desenvolvimento de software, ou seja, servem de guia para o projeto, mas o papel do cliente conhecedor das necessidades especiais e suas prefer�ncias � fundamental.
